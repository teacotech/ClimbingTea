\documentclass[a4paper,11pt]{book}
\usepackage[T1]{fontenc}
\usepackage[utf8]{inputenc}
\usepackage{lmodern}
\usepackage{hyperref}
\usepackage{graphicx}
\usepackage[english]{babel}
\newenvironment{dedication}
{
   \cleardoublepage
   \thispagestyle{empty}
   \vspace*{\stretch{1}}
   \hfill\begin{minipage}[t]{0.66\textwidth}
   \raggedright
}
{
   \end{minipage}
   \vspace*{\stretch{3}}
   \clearpage
}
\makeatletter
\renewcommand{\@chapapp}{}% Not necessary...
\newenvironment{chapquote}[2][2em]
  {\setlength{\@tempdima}{#1}%
   \def\chapquote@author{#2}%
   \parshape 1 \@tempdima \dimexpr\textwidth-2\@tempdima\relax%
   \itshape}
  {\par\normalfont\hfill--\ \chapquote@author\hspace*{\@tempdima}\par\bigskip}
\makeatother
\title{\Huge \textbf{Climbing Tea} \\ Visualization of Climbing Techniques\\}
% Author
\author{\textsc{Awwal Oppa}\thanks{\hyperref[www.google.com]{awwaloppa}}}


\begin{document}

\frontmatter
\maketitle
\begin{dedication}
	Dedicated to Mapala Panjat Tebing 
\end{dedication}
\tableofcontents
\listoffigures
\listoftables

\mainmatter
\chapter*{Preface}

\section*{Proses Instalasi}
\subsection*{Alat yang Dibutuhkan}
\begin{itemize}
	\item \textbf{Kamera} dengan resolusi maupun frame rate yang cukup tinggi (60 fps) untuk menangkap gerakan cepat.
	\item \textbf{Komputer} dengan GPU yang memadai untuk memproses model \textit{Deep Learning}.
	\item Perangkat Lunak dan Pustaka:
	\begin{enumerate}
		\item \textbf{Python};
		\item \textbf{OpenCV} untuk pemrosesan gambar;
		\item \textbf{PyTorch} atau \textbf{TensorFlow} untuk model \textit{Depth Estimation};
		\item \textbf{MiDaS} (Model \textit{Monocular Depth Estimation} dari \textit{\textbf{Intel}}).
	\end{enumerate}
\end{itemize}
\subsection*{Instalasi Perangkat Lunak}
Beberapa \textbf{dependency} atau \textbf{library} dalam konteks pengembangan perangkat lunak, paket-paket berikut ditujukan untuk menyediakan fungsi dan utilitas tertentu yang digunakan dalam pengolahan gambar, komputasi numerik, dan pengembangan model \textit{Machine Learning}:
\begin{enumerate}
	\item \textbf{Opecv-python}
	\begin{itemize}
		\item Nama Paket: opencv-python
		\item Tujuan:
			\begin{enumerate}
				\item \textbf{opencv-python} adalah pembungkus (wrapper) Python untuk \textbf{OpenCV}, yaitu pustaka yang sangat populer untuk pengolahan gambar dan visi komputer.
				\item Dengan OpenCV, dapat dilakukan berbagai operasi pengolahan gambar seperti pembacaan dan penulisan gambar, deteksi objek, transformasi gambar, pemrosesan video, dan banyak lagi.
				\item Contoh penggunaan: Membaca gambar, mendeteksi wajah dalam gambar, mengubah ukuran gambar, dll.
			\end{enumerate}
	\end{itemize}
	\item \textbf{torch}
	\begin{itemize}
		\item Nama Paket: \textbf{torch}
		\item Tujuan: 
		\begin{enumerate}
			\item \textbf{torch} adalah pustaka utama dari \textbf{PyTorch}, yang digunakan untuk komputasi numerik, khususnya dalam konteks \textit{Deep Learning}.
			\item Pytorch menyediakan tensor multidimensi (mirip dengan array di NumPy) dan alat untuk membangun dan melatih model \textit{machine learning}, termasuk model \textit{neural network}.
			\item Pytorch juga digunakan dalam penelitian AI, analisis data, dan aplikasi lain yang melibatkan \textit{machine learning}.
		\end{enumerate}
	\end{itemize}
	\item \textbf{torchvision}
	\begin{itemize}
		\item Nama Paket: \textbf{torchvision}
		\item Tujuan:
		\begin{enumerate}
			\item \textbf{torchvision} adalah pustaka yang terkait dengan PyTorch dan menyediakan fungsionalitas untuk pengolahan gambar dan visi komputer dalam konteks \textit{machine learning}.
			\item Ini mencakup dataset standar (misalnya, \textbf{CIFAR-10}, \textbf{ImageNet}), model yang telah dilatih sebelumnya (misalnya \textbf{ResNet}, \textbf{VGG}), serta alat untuk mentransformasikan gambar (misalnya, \textbf{resize}, \textbf{normalize}).
			\item Sangat berguna ketika bekerja dengan model-model visi komputer dalam \textbf{PyTorch}.
		\end{enumerate}
	\end{itemize}
	\item \textbf{numpy}
	\begin{itemize}
		\item Nama Paket: \textbf{numpy}
		\item Tujuan:
		\begin{enumerate}
			\item \textbf{numpy} adalah pustaka fundamental untuk komputasi numerik di Python.
			\item Ini menyediakan struktur data \textbf{array} multidimensi yang efisien, serta banyak fungsi untuk operasi matematika seperti statistik, aljabar linear, dan manipulasi array.
			\item \textbf{numpy} sering digunakan dalam komputasi ilmiah dan analisis data, dan menjadi dasar untuk banyak pustaka ilmiah lainnya, termasuk \textbf{PyTorch}.
		\end{enumerate}
	\end{itemize}
\end{enumerate}

\section*{Another sample section}

\section*{Structure of book}

\section*{About the companion website}
\footnote{\url{https://github.com/amberj/latex-book-template}} for this file contains:
\begin{itemize}
  \item A link to (freely downlodable) latest version of this document.
  \item Link to download LaTeX source for this document.
  \item Miscellaneous material (e.g. suggested readings etc).
\end{itemize}
\section*{Acknowledgements}
\begin{itemize}
\item A special word of thanks goes to Professor Don Knuth\footnote{\url{http://www-cs-faculty.stanford.edu/~uno/}} (for \TeX{}) and Leslie Lamport\footnote{\url{http://www.lamport.org/}} (for \LaTeX{}).
\item I'll also like to thank Gummi\footnote{\url{http://gummi.midnightcoding.org/}} developers and LaTeXila\footnote{\url{http://projects.gnome.org/latexila/}} development team for their awesome \LaTeX{} editors.
\item I'm deeply indebted my parents, colleagues and friends for their support and encouragement.
\end{itemize}

\chapter{Introductory Chapter}

\begin{chapquote}{Author's name, \textit{Source of this quote}}
``This is a quote and I don't know who said this.''
\end{chapquote}

\section{Section heading}


\subsection{}


\subsection{}

\subsection{}


\subsection{}

\section{}
\begin{table}[ht]
\caption{Sample table} % title of Table
\centering % used for centering table
\begin{tabular}{c c c c}
% centered columns (4 columns)
\hline\hline %inserts double horizontal lines
S. No. & Column\#1 & Column\#2 & Column\#3 \\ [0.5ex]
% inserts table
%heading
\hline % inserts single horizontal line
1 & 50 & 837 & 970 \\
2 & 47 & 877 & 230 \\
3 & 31 & 25 & 415 \\
4 & 35 & 144 & 2356 \\
5 & 45 & 300 & 556 \\ [1ex] % [1ex] adds vertical space
\hline %inserts single line
\end{tabular}
\label{table:nonlin} % is used to refer this table in the text
\end{table}
\begin{itemize}
\item 
\item 
\item
\end{itemize}

\subsection{}

\subsection{}

\end{document}
